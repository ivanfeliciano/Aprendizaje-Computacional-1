\begin{frame}{Trabajo relacionado}

\begin{itemize}
\item Usar la heurística ``primero el más lejano'' para saleccionar
los centroides iniciales. La complejidad del caso promedio 
sigue siendo la misma que la del K-means tradicional.

\item La aplicación de estructuras de datos de partición del espacio,
como los kd-trees, aumentan la eficiencia pero sólo para
un número pequeño de dimensiones.

\item Scalable K-means usa cada centroide como consulta para recuperar
una lista de documentos, que se son asignados a ese cluster en particular 
sin calcular las distancias. 

\end{itemize}

\end{frame}
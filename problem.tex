\begin{frame}{Descripción del problema}
    
    \begin{itemize}
        \item 
        En la actual era del ``big data'', el 
        número de documentos en la web se 
        incrementa muy rápidamente.
        \item 
        Se necesitan técnicas de gestión 
        de contenido eficientes y efectivas
        para agrupar documentos similares.
        \item
        K-means en uno de los algoritmos de agrupamiento
        más utilizados. Sin embargo, no escala bien
        para conjuntos de datos considerablemente 
        grandes, en tamaño y dimensionalidad.
    \end{itemize}
    
    
\end{frame}

\begin{frame}{Algoritmo K-means}

K-means es un algoritmo EM, en el que comenzando
con un conjunto de centroides elegidos aleatoriamente,
cada punto de entrada es asignado a su centroide 
más cercano (E). Los centroides se recalculan 
usando las asignaciones actuales (M).
\end{frame}


\begin{frame}
Poner el algoritmo, animación y complejidad $O(NPK)$
\end{frame}